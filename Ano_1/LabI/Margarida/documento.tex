\documentclass{report}
\usepackage[T1]{fontenc} % Fontes T1
\usepackage[utf8]{inputenc} % Input UTF8
\usepackage[backend=biber, style=ieee]{biblatex} % para usar bibliografia
\usepackage{csquotes}
\usepackage[portuguese]{babel} %Usar língua portuguesa
\usepackage{blindtext} % Gerar texto automaticamente
\usepackage[printonlyused]{acronym}
\usepackage{hyperref} % para autoref
\usepackage{graphicx}
\usepackage{placeins}
\bibliography{bibliografia}


\begin{document}
%%
% Definições
%
\def\titulo{Análise de redes de comunicação}
\def\data{29/11/2015}
\def\autores{Alexandre Lourenço, Susana Dias}
\def\autorescontactos{(79894) alexandre.lourenco@ua.pt, (80410) susanadias@ua.pt}
\def\versao{VERSAO 1.0}
\def\departamento{Departamento de Eletrónica, Telecomunicações e Informática}
\def\empresa{Universidade de Aveiro}
\def\logotipo{ua.pdf}
%
%%%%%% CAPA %%%%%%
%
\begin{titlepage}

\begin{center}
%
\vspace*{50mm}
%
{\Huge \titulo}\\ 
%
\vspace{10mm}
%
{\Large \empresa}\\
%
\vspace{10mm}
%
{\LARGE \autores}\\ 
%
\vspace{30mm}
%
\begin{figure}[h]
\center
\includegraphics{\logotipo}
\end{figure}
%
\vspace{30mm}
\end{center}
%
\begin{flushright}
\versao
\end{flushright}
\end{titlepage}

%%  Página de Título %%
\title{%
{\Huge\textbf{\titulo}}\\
{\Large \departamento\\ \empresa}
}
%
\author{%
    \autores \\
    \autorescontactos
}
%
\date{\data}
%
\maketitle

\pagenumbering{roman}

%%%%%% RESUMO %%%%%%
\begin{Resumo}
Este trabalho teve como objetivo analisar:
\vspace{05mm}
	
	- a variação da largura de banda em relação ao tamanho dos pacotes enviados;
	\vspace{02mm}
	- a variação do numero de pacotes recebidos em função do tamanho dos mesmos.
\end{abstract}

%%%%%% Agradecimentos %%%%%%
% Segundo glisc deveria aparecer após conclusão...
%\renewcommand{\abstractname}{Agradecimentos}
%\begin{abstract}
%Eventuais agradecimentos.
%Comentar bloco caso não existam agradecimentos a fazer.
%\end{abstract}


\tableofcontents
% \listoftables     % descomentar se necessário
% \listoffigures    % descomentar se necessário


%%%%%%%%%%%%%%%%%%%%%%%%%%%%%%%
\clearpage
\pagenumbering{arabic}

%%%%%%%%%%%%%%%%%%%%%%%%%%%%%%%%
\chapter{Introdução}
\label{chap.introducao}

Através de identificadores é possível a troca de informação em redes. Utilizando o comando ping é possível saber se o endereço de destino respondeu e quanto tempo demorou a receber a resposta. No caso de existir um erro, o endereço ping irá apresentar uma mensagem de erro. 
A partir daí foram enviados pacotes com diferentes tipos de tamanho de forma a estudar o tempo que o endereço de destino demora a responder e quanto tempo demora a receber uma resposta.

\chapter{Metodologia}
\label{chap.metodologia}

Para a obtenção dos resultados utilizamos o comando ping, enviando pacotes de diferentes tamanhos para o router de forma a perceber o tempo que este demora a responder e o tempo que demora a receber a resposta. Variando para isso a largura de banda que o comando ping
produz de forma a controlar a taxa de pacotes e o seu tamanho


\chapter{Resultados}
\label{chap.resultados}

	Através do uso do comando ping no terminal do ubunto foram registados vários testes ao router local, sendo dois deles apresentados de seguida:

%imagens dos resultados obtidos
 
\begin{figure}[!htp]
\centering
\includegraphics[scale=0.3]{print1.jpg} 
\caption{Teste de ping com envio de 250 pacotes com 32 bytes}
\end{figure}
 
\begin{figure}[!htp]
\centering
\includegraphics[scale=0.3]{print1_1.jpg} 
\caption{Teste de ping com envio de 250 pacotes com 32 bytes}
\end{figure}

\begin{figure}
\centering
\includegraphics[scale=0.3]{print2.jpg} 
\caption{Teste de ping com envio de 250 pacotes com 65000 bytes}
\end{figure}

\begin{figure}
\centering
\includegraphics[scale=0.3]{print2_1.jpg} 
\caption{Teste de ping com envio de 250 pacotes com 65000 bytes}
\end{figure}


\chapter{Análise}
\label{chap.analise}

Após a recolha dos dados resultantes dos testes de ping foi-nos possível agrupar os valores graficamente de forma a facilitar a obtenção de conclusões.
Assim sendo foram produzidos dois gráficos, em que relacionam o número de pacotes recebidos com o tamanho dos mesmos e a largura de banda, ou seja, o tempo de resposta do envio dos pacotes (neste caso 250, uma vez que convinha que todos os testes possuíssem o mesmo numero de pacotes para uma posterior comparação de valores) com o seu tamanho, respetivamente. 

\begin{figure}
\centering
\includegraphics[scale=0.3]{grafico1.jpg} 
\caption{Número de pacotes recebidos em função do tamanho}
\end{figure}
\begin{figure}
\centering
\includegraphics[scale=0.3]{grafico2.jpg} 
\caption{Largura de banda em função do tamanho}
\end{figure}

\chapter{Conclusões}
\label{chap.conclusao}

Partindo da análise dos gráficos acima apresentados, e sabendo que para todos os testes o número de pacotes enviados foi constante, pôde-se constatar que à medida que o tamanho dos pacotes enviados aumentava a largura da banda tendia para zero, chegando mesmo a atingir este valor, contudo a percentagem de pacotes perdidos era muitas vezes superior a 50\% para pacotes com valores superiores a 300 bytes. Contudo, quando se enviam pacotes mais pequenos na ordem dos 32 bytes essas perdas não se verificam, no entanto a largura de banda já ascende neste caso em particular para a casa dos 249474 ms.
Em suma, para um resultado mais viável compensa mais enviar pacotes mais pequenos de forma a evitar perdas dos mesmos. 


\chapter*{Contribuições dos autores}

\ac{al} Recolha de dados, análise e organização (50\%)

\ac{sd} Introdução, metodologia e análise (50\%)

%%%%%%%%%%%%%%%%%%%%%%%%%%%%%%%%%
%\chapter*{Acrónimos}
\begin{acronym}
\acro{ua}[UA]{Universidade de Aveiro}
\acro{al}[AL]{Alexandre Lourenço}
\acro{sd}[SD]{Susana Dias}
\acro{miect}[MIECT]{Mestrado Integrado em Engenharia de Computadores e Telematica}
\end{acronym}


%%%%%%%%%%%%%%%%%%%%%%%%%%%%%%%%%
\chapter*{Bibliografia}
\printbibliography

\end{document}
